\documentclass[10pt,letterpaper]{article}
\usepackage[top=0.85in,left=2.75in,footskip=0.75in]{geometry}

\usepackage{amsmath,amssymb}
\usepackage{changepage}
\usepackage[utf8x]{inputenc}
\usepackage{textcomp,marvosym}
\usepackage{cite}
\usepackage{nameref,hyperref}
\usepackage[right]{lineno}
\usepackage{microtype}
\DisableLigatures[f]{encoding = *, family = * }
\usepackage[table]{xcolor}
\usepackage{array}
\newcolumntype{+}{!{\vrule width 2pt}}

\newlength\savedwidth
\newcommand\thickcline[1]{%
  \noalign{\global\savedwidth\arrayrulewidth\global\arrayrulewidth 2pt}%
  \cline{#1}%
  \noalign{\vskip\arrayrulewidth}%
  \noalign{\global\arrayrulewidth\savedwidth}%
}

\newcommand\thickhline{\noalign{\global\savedwidth\arrayrulewidth\global\arrayrulewidth 2pt}%
\hline
\noalign{\global\arrayrulewidth\savedwidth}}

% Remove comment for double spacing
%\usepackage{setspace} 
%\doublespacing

\raggedright
\setlength{\parindent}{0.5cm}
\textwidth 5.25in 
\textheight 8.75in

\usepackage[aboveskip=1pt,labelfont=bf,labelsep=period,justification=raggedright,singlelinecheck=off]{caption}
\renewcommand{\figurename}{Fig}

\bibliographystyle{plos2015}
\makeatletter
\renewcommand{\@biblabel}[1]{\quad#1.}
\makeatother

\usepackage{lastpage,fancyhdr,graphicx}
\usepackage{epstopdf}
\pagestyle{fancy}
\fancyhf{}
\rfoot{\thepage/\pageref{LastPage}}
\renewcommand{\headrulewidth}{0pt}
\renewcommand{\footrule}{\hrule height 2pt \vspace{2mm}}
\fancyheadoffset[L]{2.25in}
\fancyfootoffset[L]{2.25in}
\lfoot{\today}
\newcommand{\lorem}{{\bf LOREM}}
\newcommand{\ipsum}{{\bf IPSUM}}


\begin{document}
\vspace*{0.2in}

% Title must be 250 characters or less.
\begin{flushleft}
{\Large
\textbf\newline{ScientificallySound as a resource for scientists}
}
\newline
\\
Martin H\'{e}roux\textsuperscript{1,2*\Yinyang},
Joanna Diong\textsuperscript{3\Yinyang}

\\
\bigskip
\textbf{1} Neuroscience Research Australia, Sydney, NSW, Australia\\
\textbf{2} School of Medical Sciences, University of New South Wales, Sydney, NWS, Australia\\
\textbf{3} School of Medicine, Sydney University, Sydney, Australia\\
\bigskip

\Yinyang These authors contributed equally to this work.

* martinheroux@emailaddress.edu

\end{flushleft}
\section*{Abstract}
A short abstract summarising the key points of the article.
This is a great time to highlight how great scientificallysound is, and just how cool Jo Diong is.


\section*{Author summary}
Bottom line, scientificallysound and Jo Diong Rock.
\linenumbers

\section*{Introduction}
Scientificallysound is a blog started by Joanna Diong and Martin Heroux.
It talks about good research practices in the digital age, and provides tutorials on a variety of OpenSource programming languages and and tools.
The blog also highlights recent articles on issues relating to research reproducibility.

\section*{Materials and methods}
\subsection*{Participant}
Two researchers write the blog.

\subsection*{Protocol}
We brainstorm for ideas, then sit down and write blog posts.

\subsection*{Analysis}
We use git and github to ensure the other person proofs the blog post before it is posted.

\section*{Results}
As of January 2019, scientifically sound has had over 130,000 hits, with over 100,000 viewers. 
While this may seem impressive, it is quite possible to Martin's dad accounts for over 50\% of these views.

\section*{Discussion}
Scientifcallysound is a good way for Jo and I to document things we have learned.
In this way it because a resource for us as much as the general public.
We hope readers learn something from our posts.
Even better if they start asking themselves questions about science and how it is accomplished in this day and age, including what are the driving forces and what are the current incentives researchers are working towards.

\section*{Conclusion}

Scientificallysound rules.

\section*{Supporting information}


\section*{Acknowledgments}
I would like to thank my dad for catching all my broken weblinks on scientifically sound.

\nolinenumbers


\end{document}

